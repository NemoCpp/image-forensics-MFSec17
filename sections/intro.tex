\section{INTRODUCTION}
%% TODO: introduce Content Analysis and role in DARPA MEMEX
Over the past two years, our research team has borne witness to the ease and availability of potentially criminal goods and services on the modern Internet. In particular, our team's work on the DARPA Memex project has focused on the issues of online gun sales, as such sales can have grim consequences in that they provide a medium for buyers and sellers to circumvent traditional background checks. In turn, this proliferates the sale of dangerous semi-automatic weapons and can lead directly to loss of human life. For instance, a New York Police Department (NYPD) investigation in 2013 identified guns used in one suicide and four murders and traced their origin to transactions on the website \url{armslist.com}\cite{raja_2016}. 

The ability to rapidly detect and analyze gun sales transactions is a significant challenge. There are hundreds of both national and regional gun sales sites like Armslist, \url{floridaguntrader.com}, or \url{gunbroker.com}. Besides the sheer number of sites, many of the sites share common themes indicative of today's {\em Deep} web. They require a login to either buy or sell, making bulk analysis difficult for traditional web crawlers. A significant number of the sites use Ajax or Javascript for pagination, or for displaying gun images from an ad; this also makes automatic analysis a challenge. But most significantly, the actual content required to determine the answers to significant questions regarding these weapons (``is this an automatic weapon?'', ``is this a long or a short gun?'', ``are there multiple weapons being sold?'') are the {\em images} of the weapons themselves. 

We have previously worked on bulk image analysis from the deep web as it relates to human trafficking data \cite{mattmann7tg} using the Apache Tika content detection and analysis framework \cite{mattmann2011tika}. Our work there focused on image metadata forensics as an alternative to image pixel based analyses and object detection and recognition. Though metadata forensics were promising in human trafficking, weapons required pixel-based analyses. Based on our study of over 80 websites and online forums that specialize in the exchange of weapons, object recognition and computer vision were needed to automatically discern whether or not the guns being sold are automatic or semi-automatic, whether they have been stolen (using serial-number identification), and whether the transactions are potentially illegal. Automatically being able to discern these types of object properties in bulk analyses of image data has the potential to thwart crimes and, ultimately, to save lives.

Law enforcement agencies do not have the manpower to effectively monitor the scale of weapons ads on the aforementioned sites and forums and though the ads, like the whole Internet, contain a large amount of text \cite{mphillips-EOT2012}, the proliferation of images necessitates object recognition and image analysis at scale. Our recent work in DARPA's MEMEX initiative has focused on expanding Apache Tika to support such analysis.

Historically, the best object recognition systems were inaccurate, but this has changed due to recent advancements in deep neural networks, larger training datasets, and improved computing resources. Tensorflow is a scalable, Python-based system and it natively supports image recognition via its \em {Inception} model \cite{abadi2016tensorflow}. \em{Inception} provides a neural network trained on the ImageNet corpus \cite{krizhevsky2012imagenet}, a dataset of 14,197,122 images classified using text from the WordNet taxonomy. The end result is a highly-scalable off-the-shelf system that can accurately identify and classify objects in images into a thousand categories. Combining this capability with Apache Tika's native support for detecting thousands of file formats, extracting their metadata and textual content, is an attractive, automated solution that can perform bulk analysis in the weapons domain, but more generally, in any context where text and images are present and such analyses are required.

Integration of Tensorflow with Tika presented a significant challenge: Tensorflow does not provide out of the box bindings to Java based frameworks. Apache Tika is primarily written in Java and thus integrating with Tensorflow is not straight forward like any other JVM compatible libraries. Our research directly addresses this and contributes several methods that make Tensorflow easier to integrate into Java-based systems like Tika, and any digital forensics system that can make a call to an application programming interface (API). In this paper, we report on our integration of Tika and Tensorflow using the Weapons domain as a motivating example. We evaluate the integration in both its robustness in objection recognition with zero training beyond that of ImageNet and demonstrate that Tensorflow and Tika are a scalable forensics solution for bulk image analysis on the Deep Web.

The remainder of this paper is organized as follows. Section 2 discusses the collection of the Weapons dataset and its properties. Section 3 presents our integration of Tika and Tensorflow via three methods: (1) command line invocation; (2) Google's RPC (gRPC) integration; and (3) Representational Entity State Transfer (REST) \cite{Fielding:2000:ASD:932295} integration. Section 4 qualitatively evaluates our integration techniques and quantitatively evaluates the efficacy of Tensorflow, ImageNet and Tika-based image forensics. Section 5 rounds out the paper.